\documentclass[10pt, a4paper]{article}

\usepackage[a4paper, top=0.5cm, bottom=0.5cm, left=0.5cm, right=0.5cm, landscape]{geometry}
\usepackage{mathtools}
\usepackage{amsfonts}
\usepackage{multicol}
\usepackage{setspace}
\usepackage{graphicx}
\usepackage{xcolor}



\author{Zachary Chua Yan Ern}
\date{22 September 2021}
\setstretch{1.25}

\newcommand{\highlight}[1]{{\color{red}\textbf{#1}}}

\begin{document}
	\scriptsize %tiny
	\setlength\parindent{0pt}
	\setlength{\columnseprule}{0.1pt}
	
	\begin{center}
		{\large CS2105 CheatSheet}\\
		by Zachary Chua
	\end{center}
	
	\begin{multicols*}{3}
		{\normalsize\textbf{Internet}}\\
		- The Internet is a network of connected computing devices\\
		- Devices are known as \highlight{hosts} or \highlight{end systems}\\
		- Hosts run network applications (like browsers) and communicate over links\\
		
		\textbf{Network Edge (Access Network)}\\
		- Hosts access the Internet through \highlight{access network}\\
		- eg Home/ Institute access networks\\
		
		\textbf{Wireless Access Network}\\
		1. Wireless LAN (WIFI): Short range (100 ft)\\
		2. Wide-area wireless access (3G/ 4G): Long range (10s km)\\	
		
		\textbf{Physical Media}\\
		Host connect to access network via physical media
		- Guided media: signals propogate in solid media (ethernet cable/ fibre optics)\\
		- Unguided media: signals propagate freely (radio)\\
		
		\textbf{Network Core}\\
		A mesh of interconnected routers\\
		Data transmitted by\\
		1. Circuit Switching: dedicated circuit per call\\
		2. Packet Switching: data sent through net in discrete "chunks"\\
		
		\textbf{Circuit Switching}\\
		End-to-end resources \highlight{allocated to and reserved for} "call" between source and dest\\
		- call setup required\\
		- circuit-like (\highlight{guaranteed}) performance\\
		- circuit segment idle if not used by call\\
		- used in traditional telephone networks\\
		- limited capacity\\
		
		\textbf{Packet Switching}\\
		Host sending function\\
		- breaks application message into smaller chunks, known as \highlight{packets} of length \highlight{L} bits\\
		- transmits packets onto the link at \highlight{transmission rate R}\\
		- link transmission rate is known as \highlight{link capacity} or \highlight{link bandwidth}\\
		
		Packet Transmission Delay = $\frac{L}{R}$, assuming packet size $L$ bits and link bandwidth $R$ bits/sec\\
		
		\highlight{Store and Forward}: entire packet must arrive at a router before it can be transmitted on the next link (check packet integrity; if corrupted, drop packet)\\
		
		\textbf{Routing and Addressing}\\
		
		
		
		
		
		
	\end{multicols*}
\end{document}